\section{Relatório sobre
as Áreas da Edra}



A Edra é uma equipe de competição criada em 2017 por estudantes de engenharia da Universidade de Brasília, com um experiencia no ramo de produção de drones para competição, tendo em vista a extrema complexidade desse trabalho, se vê necessário a divisão das tarefas em áreas com funções especificas, sendo elas:

\subsection{Design Estrutural}



A área de design estrutural é responsável pelo frame do drone, a estrutura base, onde irão fixados todo os demais componentes, (fazendo uma analogia com o corpo humano, seria o esqueleto).\\[0.2cm]



Durante uma decolagem, voo e pouso, um drone sofre diversos esforços,

o frame do drone deve ser capaz de suportar todos esses esforços sem danificar nenhum componente, para isso é necessário o trabalho de modelagem e simulação de forças.\\[0.2cm]



O trabalho de modelagem consiste em criar peças em 3D visando atender as necessidades estruturais do drone, esse processo é feito através de softwares CAD, como solidWorks, Cátia e Fusion 360.
\\[0.2cm]


Em geral, os drones da Edra são quadrirrotores, sendo assim suas estruturas consistem em 4 hastes onde irão os motores e uma placa central para os eletrônicos.
\\[0.2cm]


É necessário validar o frame via simulação de esforços, esse processo consiste em virtualmente aplicar uma força determinada em um conjunto de peças ou em uma peça especifica e analisar a maneira que a mesma reage.

  \\[0.2cm]

\subsection{Aerodinâmica e propulsão}

A área de Aerodinâmica e propulsão é responsável por todo o sistema propulsivo do drone, todo o conjunto encarregado de gerar empuxo para a o voo do drone. \\[0.2cm]

Diversos fatores são responsáveis pela geração de empuxo em um drone, alguns deles são: \\[0.2cm]

O sistema motor hélice, o perfil de aerofólio da hélice e a potencia do motor interferem diretamente na capacidade de empuxo que esse conjunto pode gerar. \\[0.2cm]

O conjunto bateria e esc., esc. é a placa controladora de velocidade do motor, ela é responsável por regular a potencia do motor, a bateria é a encarregada por alimentar todo esse sistema com energia, a capacidade energética da bateria influencia diretamente no tempo de voo de um drone. \\[0.2cm]

Dessa maneira, a área de aero propulsão é a responsável por escolher todo esses componentes minuciosamente e realizar testes com os mesmos. \\[0.2cm]



\subsection{Eletro - software}



Eletro Software é a fusão entre duas áreas, software e eletrônica.  \\[0.2cm]

A Area de Software é responsável por desenvolver aplicações para o drone, por exemplo, mapeamento aéreo e reconhecimento de objetos. Usando ferramentas como Python, uma linguagem de programação de auto nível, OpenCV, uma multiplataforma para desenvolvimento de visão computacional e Tensor Flow, uma biblioteca de código aberto para aprendizado de máquina. \\[0.2cm]



A área de Eletrônica é responsável pelo radio controlador, instrumento usado para comunicação do piloto com o drone e todo o sistema de periféricos, isso engloba os módulos de telemetria,  uma tecnologia que permite a medição e comunicação de informações, modulo GPS, usado pelo drone para se localizar no espaço, uma gama de sensores, como câmera, acelerômetro, dentre outros. \\[0.2cm]

O conhecimento técnico sobre leitura de datasheets e componentes eletrônicos é vital para o bom funcionamento da área, assim como a habilidade de soldagens de componentes com estanho. \\[0.2cm]





\subsection{Controle e Estabilidade}

A Area de controle é responsável por desenvolver e aprimorar o sistema embarcado de controle do drone, isto é feito através de modelos matemáticos para prever a maneira de comportamento da aeronave, desse modo é simulado matematicamente e realizado teste para analisar o desempenho do físico do drone em relação ao simulado matemático. \\[0.2cm]



\subsection{Integração entre as Áreas}



Apesar da enorme diferença entre funções e tarefas de cada Area da Edra, todas possuem uma enorme comunicação e interdependência. \\[0.2cm]

A comunicação entre as áreas ocorre de diversas maneiras, todas as áreas de correlacionam no drone como um todo, entretanto alguns sistemas e subsistemas estão intrinsicamente ligados, como por exemplo: \\[0.2cm]

A área de software depende das escolhas de componentes feitos pela área de eletrônica, caso a área de software deseje realizar um reconhecimento de objeto com base em imagens, é necessário que o sensor ótico cumpra com as necessidades da área de software. \\[0.2cm]

A área de aerodinâmica e propulsão depende da área de estrutura, caso a área de aero propulsão escolha um determinado conjunto motor hélice e bateria, é necessário que o frame do drone possua o espaço física necessário para a acoplagem desses componentes.
\\[0.2cm]


Na engenharia Aeroespacial, todo os sistemas estão interligados, isto fica muito evidente da Edra.\\[0.2cm]




