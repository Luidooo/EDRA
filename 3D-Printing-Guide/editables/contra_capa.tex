\textbf{\Large Autor:}\\[0.6 cm]
\begin{center}
    \textbf{\medium Luis Eduardo Castro Mendes de Lima}\\[0.3cm]
    \textbf{\Large 221008285}\\[1.0cm]
\end{center}
\textbf{\Large Resumo:}\\[0.6cm]

    Tendo em vista a necessidade da Equipe de Robótica Area (EDRA) da Universidade de Brasília (UnB) e o interesse do estudante de engenharia Luis Eduardo Castro Mendes de Lima foi proposto um processo seletivo extraordinário. Este procedimento de seleção visa avaliar o aluno para possivelmente o mesmo ingressar na equipe de competição e capacitar o aluno em sua área de interesse, manufatura aditiva para a indústria aeroespacial.\\[0.6cm]

    Desse modo foi solicitado ao aluno em estágio de trainee realizasse um documento por escrito com os seguintes conteúdos:\\[0.4cm]

    Um manual de uso para impressoras 3D, explicando desde da concepção da peça até os processos necessários para a impressão.\\[0.4cm]

    Um estudo teórico sobre filamentos para impressão 3D, a análise das propriedades mecânicas e químicas dos materiais utilizados para a manufatura aditiva.\\[0.4cm]

    Um relatório sobre as áreas da EDRA e suas interligações.\\[0.4cm]

