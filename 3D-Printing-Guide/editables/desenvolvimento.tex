\section{Estudo teórico de filamentos para impressão 3D}
\\[0.4cm]


\subsection{Contextualização} \\[0.2cm]



É intrínseco do processo de produção a necessidade de matéria prima. A manufatura aditiva através de fusão de filamento depositado, a impressão 3D, usa como material base os polímeros termoplásticos, que basicamente são polímeros que a certa temperatura se tornam viscosos("derretem") e ao serem \\[0.2cm] resfriados a temperatura ambiente, se tornam sólidos novamente.\\[0.2cm]

Em geral, o material para impressão 3D, popularmente chamado de filamento, é vendido em quilos, em formato de fio, de 1,75 mm de diâmetro, enrolado em um carretel.\\[0.2cm]





\subsection{impressão 3D de drones}\\[0.2cm]



A manufatura aditiva possibilita a produção de peças com geometrias complexas, com o preenchimento interno de padrões otimizados e a prototipagem consideravelmente rápida. Tendo em vista essas caracterizas, a impressão 3d se torna uma forte aliada para a criação de drones para competição, pois é possível confeccionar peças que atendem as necessidades de otimização dos drones. \\[0.2cm]



Com o surgimento de novos filamentos com propriedades especificas, o uso de impressão  3D para o mundo dos drones se tornou ainda mais comum.\\[0.2cm]



\subsection{Os clássicos} \\[0.2cm]



Por volta de 2017, houve uma forte populariza-o de impressoras para o mercado residencial. Essas impressoras foram constituídas para serem capaz de processarem os 3 principais tipos de filamento do mercado, desse modo, o mercado se limitou a filamentos específicos durante alguns anos. \\[0.2cm]

Tais filamentos eram, Poli Ácido Lático(PLA), Acrilonitrila Butadieno Estireno (ABS) e Poli Estireno Tereftalato com Glicol(PETG).\\[0.2cm]





\subssubection{Poli Acido Lático(PLA)}



O Poliácido láctico ou Acido Polilactitco, tem sua origem a partir da repetição de diversas cadeias químicas de acido lático, que é obtido a partir da fermentação de alimentos ricos em amido.\\[0.2cm]

Nas condições corretas pode ser Biodegradável. \\[0.2cm]



Amplamente usado para impressão pela sua facilidade de manuseio\\[0.2cm] .



Com densidade de 1,27 g/cm³, resistência à Flexão entre 65 – 75 MPa, resistência à Tração entre 55 - 65 MPa e temperatura máxima de trabalho de 50 graus. O PLA possui uso mais voltado a peças sem grandes esforços, peças estéticas ou de acabamento,  por exemplo, uma de suas principais vantagens é a alta dureza, sendo usado também para peças e objetos que não podem sofre deformação e sua relativa baixa temperatura máxima de trabalho o impede de ser usado para aplicações externas ou com elevado atrito. \\[0.2cm]



Uma das vantagens do Pla é seu custo beneficio, sendo vendido por cerca de 100 reais o quilo.\\[0.2cm]







\subsubsection{Acrilonitrila Butadieno Estireno (ABS)}





O ABS é um copolímero composto pela fusão de acrilonitrila, butadieno e estireno, a proporção de cada componente pode variar de 15\% a 35\% de acrilonitrila e 40\% a 60\% de estireno, com 5\% a 30\% de butadieno, formando assim, uma cadeia longa de polibutadiaeno com interligações de cadeias curtas de estireno e acrilonitrila. \\[0.2cm]



Com densidade de 1,05 g/cm³, resistência à Flexão de 77 MPa, resistência à Tração de 46 MPa e Temperatura de Deflexão \\[0.2cm]

Térmica de 88 °C. Por possuir elevada temperatura máxima de trabalho, o abs. possui aplicações voltadas a peças que estarão em ambientes externos, entretanto, sua relativa ductibilidade pode ocasionar em pequenas deformações caso sofra esforços elevados.
\\[0.2cm]


O ABS possui uma contração muito elevada, sendo assim, um material de alta dificuldade para impressão, pois ele tende a descolar da mesa de impressão caso resfrie rapidamente, desse modo, sendo necessário impressoras de ambiente de impressão fechadas para realizar seu processamento.\\[0.2cm]



Com custo um pouco inferior ao PLA, o abs. é comercializado por cerca de 85 reais o quilo. \\[0.2cm]





\subsubsection{Poli Estireno Tereftalato com Glicol(PETG)}



O PetG é uma junção de polímeros formados à parir da reação do terá ftálico e etileno Glicol, tem sua origem à partir da derivados da produção de petróleo e de gás natural.\\[0.2cm]





Com densidade de 1,26 g/cm³, resistência à flexão de 65 MPa, resistência à tração de 48 MPa e temperatura de deflexão .\\[0.2cm]

térmica de 72 °C. O Petg é o intermédio entre o ABS e o PLA, não é tão dúctil quanto o ABS e nem tão duro quanto o PLA e possui uma temperatura de trabalho relativamente alta, sendo assim, sua aplicação principal é para peças externas ou internas que sofreram algum tipo de esforço mecânico mas que não podem sofrer deformação em excesso. \\[0.2cm]



Certa facilidade para impressão, é recomendável impressora de ambiente de impressão fechado mas não é algo imprensincidivel.\\[0.2cm]



Com custo mais elevado que o PLA e ABS o petg é comercializado por cerca de 120 reais o quilo. \\[0.2cm]



\subsection{Os novos 4} \\[0.2cm]



Com o amadurecimento do mercado de impressão 3D, os usuários sentiram necessidade de filamentos com características mecânicas e químicas melhoradas, naturalmente essa necessidade foi a entidade pelas empresas vendedoras de material para impressão.\\[0.2cm]

Desse modo, houve a popularização de 4 novos tipos de filamentos, O Termoplástico Poliuretano(TPU), O Poliácido de Metileno-Acetal(POM), A poliamida ou Nylon e o PetG com adicional de Fibras de Carbono.\\[0.2cm]





\subsection{PetG com Adicional de Fibras de Carbono}

Materiais compósitos são matérias que separados possuem propriedades diferentes mas quando juntos melhoram as suas características, sempre são formados por uma base e um reforço, nesse caso a base é o Petg e o reforço são partículas de fibras de Carbono.\\[0.2cm]



Com densidade de 1,29 g/cm³, resistência à flexão de 93 MPa, resistência à tração de 66 MPa e temperatura de deflexão.\\[0.2cm]

térmica de 90 °C. O petg com adicional de fibra de carbono é recomendável para peças com grandes esforços mecânicos, sendo um material muito importante para a o mundo aeroespacial, levando em consideração sua relativa baixa densidade e alta resistência.\\[0.2cm]



Por ter como base o PetG, possui a mesma relativa facilidade de impressão, entretanto um material muito mais abrasivo, possivelmente sendo necessário a troca do bico extrusor por abrasão. \\[0.2cm]



De custo relativamente elevado, o PetG com adicional de fibra de Carbono por é comercializado por cerca de 220 reais 800 gramas.\\[0.2cm]



\subsection{Termoplástico Poliuretano(TPU)}



O termoplástico Poliuretano consiste na repetição de diversas cadeias químicas de uretanos, tendo sua produção com base em derivados do petróleo. Com características de deformação interessantes o poliuretano já um velho conhecido da indústria plástica. \\[0.2cm]



Com densidade de 1,22 g/cm³, resistência à flexão de 14 MPa, resistência à tração de 50 MPa e  incríveis 550\% de alongamento a ruptura. O TPU  possui amplo uso para peças com altíssimas deformações e resistência a impacto. \\[0.2cm]



De certa facilidade de impressão, se recomenda desabilitar o parâmetro de retração caso seja utilizado por impressoras com extrusoras do tipo Bowlden. \\[0.2cm]



Tem seu valor comercial por volta de 200 reais o quilo. \\[0.2cm]





\subsection{Poliácido de metileno-acetal(POM)}



Poliácido de metileno-aceta ou Poliacetal,provém do formol de aldeído e grupos de

acetato. Por causa de suas características químicas, semicristalino, o faz ser branco e opaco. Por ter características de baixo atrito, possui amplo uso na industrial plástica de injeção. \\[0.2cm]



Com um Índice de Fluidez de 9 ±1 g/10min, resistência à flexão de 85 MPa e tensão de escoamento à Tração de 62 MPa. O POM possui amplo uso para peças com necessidade de baixa fricção, boa estabilidade dimensional e elevada rigidez, engrenagens e conjuntos mecânicos por exemplo. \\[0.2cm]



Alta dificuldade para aderir a mesa de impressão, certa necessidade de conhecimento técnico e testes com o maquinário.\\[0.2cm]



Com custo de mercado de cerca de 180 reais o quilo. \\[0.2cm]



\subsection{Nylon}



Nylon ou Poliamida é um composto polimérico formado por ligações peptídicas de amidas, logo apos sua criação teve amplo uso como fibra sintética entretanto logo passou a indústria plástica pelas suas propriedades. \\[0.2cm]



Com densidade relativa de 1,157 g/cm³, resistência à flexão de 84,9 MPa, resistência à tração de 64,6 MPa e temperatura de deflexão .\\[0.2cm]

térmica de 75 °C. O nylon é um excelente filamento para peças que buscam flexibilidade e resistência. \\[0.2cm]



Um filamento extremamente hidrofílico, requer armazenamento em local seco e de relativa facilidade de uso.\\[0.2cm]



De custo beneficio alto o Nylon é vendido por volta de 200 reais o quilo. \\[0.2cm]






